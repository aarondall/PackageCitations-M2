\documentclass[11pt]{article}
% \usepackage{cite}
\usepackage{hyperref}
\begin{document}

\title{Citing \texttt{Macaulay2} Packages}
\author{Aaron Dall}
\date{\today}
\maketitle

The \texttt{PackageCitations}~\cite{PackageCitationsSource} package for \texttt{Macaulay2}~\cite{M2} gives the user a quick method for producing a \texttt{bibtex} entry of any Macaulay2 package for inclusion in the bibliography of a \LaTeX~document. The \texttt{cite} method works ``out of the box'' on all packages with the following caveats and exceptions.

\begin{enumerate}
	\item \textbf{No Authors}: The packages \texttt{Style}~\cite{StyleSource} and \texttt{Text}~\cite{TextSource} have no credited authors and so, in turn, neither do their references. Applying \texttt{cite} to either of these packages yields a citation with an empty \texttt{author} tag and a printed warning similar~to
	\begin{verbatim}
	Warning: The "Text" package provides insufficient cit-
	ation data: author.
	\end{verbatim}
	Package writers are encouraged to give data for at least one author of the package.

	\item \textbf{Authors and Contributors}: There are a number of packages that slightly abuse the \texttt{Authors} key in their option table by listing certain people as ``contributors''. The \texttt{cite} method ignores contributing authors for the time being. See, e.g.,~\cite{VisualizeSource,GraphsSource,GraphicsSource,PHCpackSource}. It might be a good idea to encourage package writers to only list authors that should appear on the citation in the option table and to credit contributors in the documentation of the package.

	\item \textbf{Titles for source code}: For a package \texttt{Foo}, the \texttt{cite} method attempts to make a reasonable title from the package name and headline. We call the headline \texttt{H} of \texttt{Foo} \emph{good} if it satisfies each of the following properties:
		\begin{itemize}
			\item \texttt{H} consists of at least one word and at most ten words;
			\item \texttt{H} does not contain a colon;
			\item \texttt{H} is not \texttt{Foo} nor some slight variation of it.
		\end{itemize}
	If \texttt{H} satisfies these properties then the title given in the citation will be \texttt{Foo: H}. See, e.g.,~\cite{PackageCitationsSource}. Otherwise the title will be the more generic \texttt{Foo: A} \emph{Macaulay2} \texttt{package}. See, e.g.,~\cite{PieriMapsSource}. The \texttt{cite} method can handle quotes (see, e.g.,~\cite{BrunsSource}) and will transform any occurrence of the string ``Macaulay2'' into $\emph{Macaulay2}$. Moreover, the tex string in~\cite{HodgeIntegralsArticle} is handled properly. However, package writers might be encouraged to avoid using quotes and tex strings in their package headlines as a general rule.

	\item \textbf{No link to source}: It is possible that, by oversight or policy, an external \texttt{Macaulay2} package \texttt{Foo} can provide no link to its source code. See, e.g.,~\cite{MatroidsSource}. In this case \texttt{cite} returns a partial citation and issues the warning:
	\begin{verbatim}
	Warning: The "Foo" package provides insufficient citation
	data: howpublished.
	\end{verbatim}

	\item \textbf{Certified Packages}: For \emph{certified} Macaulay2 packages, the method \texttt{cite} gives two citations: one for the version of the source code being used, and one for the journal article certifying the package. See, for example,~\cite{QuillenSuslinSource,QuillenSuslinArticle} or~\cite{PolyhedraSource,PolyhedraArticle}.

	\item \textbf{Diacritics}: The following collection of diacritics are handled properly by the method \texttt{cite}: \'a, \aa, \ae, \`e, \'e, \`o, \o. See, e.g.,~\cite{PolyhedraSource,LatticePolytopesSource,Schubert2Source}. These are all that are needed for \texttt{cite} to work with the \texttt{Macaulay2} packages presently available but may need to be extended in the future.

	\item \textbf{Suspected Typos}: The \texttt{cite} method makes no attempts to rectify typographical errors in the data provided by the source code. The user has been warned. For possible typos in titles see, e.g.,~\cite{BrunsSource,HigherCIOperatorsSource,MarkovSource}. For possible typos in version numbers see~\cite{DivisorSource}.

\end{enumerate}


\nocite{*}
\bibliography{testCite}{}
\bibliographystyle{plain}
\end{document}